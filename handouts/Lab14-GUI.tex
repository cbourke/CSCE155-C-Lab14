\documentclass[12pt]{scrartcl}


\setlength{\parindent}{0pt}
\setlength{\parskip}{.25cm}

\usepackage{graphicx}

\usepackage{xcolor}

\definecolor{darkred}{rgb}{0.5,0,0}
\definecolor{darkgreen}{rgb}{0,0.5,0}
\usepackage{hyperref}
\hypersetup{
  letterpaper,
  colorlinks,
  linkcolor=red,
  citecolor=darkgreen,
  menucolor=darkred,
  urlcolor=blue,
  pdfpagemode=none,
  pdfkeywords={}
}

\definecolor{MyDarkBlue}{rgb}{0,0.08,0.45}
\definecolor{MyDarkRed}{rgb}{0.45,0.08,0}
\definecolor{MyDarkGreen}{rgb}{0.08,0.45,0.08}

\definecolor{mintedBackground}{rgb}{0.95,0.95,0.95}
\definecolor{mintedInlineBackground}{rgb}{.90,.90,1}

%\usepackage{newfloat}
\usepackage[newfloat=true]{minted}
\setminted{mathescape,
               linenos,
               autogobble,
               frame=none,
               framesep=2mm,
               framerule=0.4pt,
               %label=foo,
               xleftmargin=2em,
               xrightmargin=0em,
               startinline=true,  %PHP only, allow it to omit the PHP Tags *** with this option, variables using dollar sign in comments are treated as latex math
               numbersep=10pt, %gap between line numbers and start of line
               style=default, %syntax highlighting style, default is "default"
               			    %gallery: http://help.farbox.com/pygments.html
			    	    %list available: pygmentize -L styles
               bgcolor=mintedBackground} %prevents breaking across pages
               
\setmintedinline{bgcolor={mintedBackground}}
\setminted[text]{bgcolor={mintedBackground},linenos=false,autogobble,xleftmargin=1em}
%\setminted[php]{bgcolor=mintedBackgroundPHP} %startinline=True}
\SetupFloatingEnvironment{listing}{name=Code Sample}
\SetupFloatingEnvironment{listing}{listname=List of Code Samples}


\title{CSCE 155 - C}
\subtitle{Lab 14.0 - Graphical User Interface Programming}
\author{Dr.\ Chris Bourke}
\date{~}

\begin{document}

\maketitle

\section*{Prior to Lab}

Before attending this lab:
\begin{enumerate}
  \item Read and familiarize yourself with this handout.
\end{enumerate}

Some additional resources that may help with this lab:
\begin{itemize}
  \item GTK tutorial: \url{http://developer.gnome.org/gtk-tutorial/stable/}
  \item Another GTK Tutorial: \\
  \url{http://www.yolinux.com/TUTORIALS/GTK+ProgrammingTips.html#BASICCONFIGURATION}
  \item GTK API: \url{https://developer.gnome.org/gtk3/}
  \item Possible solution for compiling, running on Windows: \\
  \url{http://stackoverflow.com/questions/1450445/}
\end{itemize}

\section*{Peer Programming Pair-Up}

To encourage collaboration and a team environment, labs will be
structured in a \emph{pair programming} setup.  At the start of
each lab, you will be randomly paired up with another student 
(conflicts such as absences will be dealt with by the lab instructor).
One of you will be designated the \emph{driver} and the other
the \emph{navigator}.  

The navigator will be responsible for reading the instructions and
telling the driver what to do next.  The driver will be in charge of the
keyboard and workstation.  Both driver and navigator are responsible
for suggesting fixes and solutions together.  Neither the navigator
nor the driver is ``in charge.''  Beyond your immediate pairing, you
are encouraged to help and interact and with other pairs in the lab.

Each week you should alternate: if you were a driver last week, 
be a navigator next, etc.  Resolve any issues (you were both drivers
last week) within your pair.  Ask the lab instructor to resolve issues
only when you cannot come to a consensus.  

Because of the peer programming setup of labs, it is absolutely 
essential that you complete any pre-lab activities and familiarize
yourself with the handouts prior to coming to lab.  Failure to do
so will negatively impact your ability to collaborate and work with 
others which may mean that you will not be able to complete the
lab.  

\section{Lab Objectives \& Topics}
At the end of this lab you should be familiar with the following
\begin{itemize}
  \item Graphical User Interface and Event-based Programming in C using GTK
  \item Compiling and running a GUI program
\end{itemize}

\section{Background}

Many programs interact with users using a Graphical User 
Interface (GUI).  Traditional GUI design involves the creation 
and interaction of widgets: general graphical elements that 
include labels, text boxes, buttons, etc.  Most GUI programming 
is done using an Application Programmer Interface (API) 
which is a library or framework that provides a lot of the core 
functionality including:
\begin{itemize}
  \item A window manager that handles the interaction of widgets
  \item A windowing system that works with the underlying operating 
  	system and hardware to render the graphics
  \item Factory functions that can be used to construct and configure widgets
\end{itemize}
The particular API that we will work with is GTK (GNU Tool Kit) which 
is written in C and has support for linux and windows systems.  
However, given the configuration issues we will focus on linux.  

For this lab, be sure to login to your CSE account through the linux system in the labs.

\subsection*{A simple GUI program}

We have provided a simple calculator program, (\mintinline{text}{gtk_calculator.c}) 
that simulates a simple calculator.  Two editable input boxes 
and one non-editable output box have been implemented 
along with several buttons to support arithmetic operations 
(addition, subtraction, multiplication).  The operations work 
by grabbing the values in the two input boxes, performing 
the arithmetic operation and placing the result into the output 
box.

You will extend the functionality of this program by adding 
another button to support division.  To do this, you will need 
to add code to create and configure the button, implement 
its callback function, and demonstrate the running program 
to your lab instructor.

\section{Activities}

Clone the project code for this lab from GitHub using the following
URL: \url{https://github.com/cbourke/CSCE155-C-Lab14}.

\subsection{Getting Familiar with Linux, GTK}

For this lab you will need to use the SUSE Linux operating system.  
All the lab computers are dual-boot.  Follow the instructions to login 
to the SUSE partition on your computer.

\begin{enumerate}
  \item Login to your account using Linux
  \begin{enumerate}
    \item Press Alt-Control-Delete as if you were going to login.
    \item Click the ``restart'' red button at the bottom right and 
    	restart the computer
    \item When it restarts, an option appears: select openSUSE
    \item Login with your cse login credentials
    \item You can open a terminal or multiple terminals (which is very 
    	much like putty) by clicking the terminal icon at the bottom left
    \item You can edit your code using the usual utilities or use a 
    	WISIWYG editor called gvim by entering gvim at the command line; 
	example: \mintinline{text}{gvim gtk_calculator.c}
  \end{enumerate}
  \item Build the application by typing \mintinline{text}{make} and 
  	run the resulting executable.  Try the program out on several 
	values to get a feel for how it works.
  \item Open the source file and look it over to get an idea of its design 
  	and structure.
  \item Close the calculator program and answer the questions from 
  	your worksheet.
\end{enumerate}
	
\subsection{Modifying the Program}

You will now modify this program to support division.

\begin{enumerate}
  \item Create a new division button:
  \begin{enumerate}  
    \item In the \mintinline{c}{main} function, declare a \mintinline{c}{GtkWidget} pointer 
	for the division button
    \item Set the pointer to a new button by calling the \mintinline{c}{gtk_button_new_with_label} function
    \item Make sure that the button is visible by calling the \mintinline{c}{gtk_widget_show} function 
    \item Add the new button to the layout by using the \mintinline{c}{gtk_box_pack_start} function
  \end{enumerate}
  \item Make the new button functional by defining and associating an 
  	appropriate callback function:
  \begin{enumerate}
    \item Create a new function (call it \mintinline{c}{division}; there is 
    	already a function called \mintinline{c}{div} that you should not 
	redefine) with the appropriate parameters.  For a function to be 
	used as a GTK callback, it must have a specific signature (return 
	type, parameter type and list)
    \item In this function, compute the division of the two input boxes and 
    	place the result into the output box.
    \begin{itemize}
      \item Utilize the \mintinline{c}{getInput} helper function we have provided
      \item Text boxes can only handle strings, so you'll need to use \mintinline{c}{sprintf} to convert the numerical result to a string
      \item Use the \mintinline{c}{gtk_entry_set_text} function to set the text of the output text box
      \item Use the other methods as examples for how to accomplish all of this
    \end{itemize}
    \item Register (bind) your division function to your division button by using the \mintinline{c}{g_signal_connect} function
  \end{enumerate}
  \item Compile your program (using \mintinline{text}{make}), run it on several 
  	values to make sure it works; demonstrate your working version to a lab instructor.
\end{enumerate}

\section{Advanced Activity (optional)}

Test how your program reacts to certain ``bad'' inputs: if you enter 
non-numeric characters in the input, or attempt to divide by zero.  
How might you prevent and/or handle these bad inputs?  One 
solution is to simply display \textbf{\textcolor{red}{ERROR}} (red, 
bold) in the output box.  Use the ``Calculator!'' label as a model 
and read GTK?s API Documentation on styling to understand 
how text elements are ``decorated''.  Add additional code to 
check for division by zero and instead of placing an answer in 
the output box, place \textbf{\textcolor{red}{ERROR}} in red, bold font.

	
\end{document}
